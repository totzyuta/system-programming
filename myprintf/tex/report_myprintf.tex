\documentclass[a4j]{jarticle}

\usepackage{listings}

\lstset{%
  language={},
  basicstyle={\small\ttfamily},%
  identifierstyle={\small},%
  commentstyle={\small\itshape},%
  keywordstyle={\small\bfseries},%
  ndkeywordstyle={\small},%
  stringstyle={\small\ttfamily},
  frame={tb},
  breaklines=true,
  columns=[l]{fullflexible},%
  numbers=left,%
  xrightmargin=0zw,%
  xleftmargin=3zw,%
  numberstyle={\scriptsize},%
  stepnumber=1,
  numbersep=1zw,%
  lineskip=-0.5ex%
}

\title{システムプログラミング\\期末レポート}
\author{\\学籍番号:09425566\\氏名:佐藤 佑太}
\date{出題日:2014/12/24\\提出日:2014/1/22\\締切り日:2014/1/22\\}

\begin{document}

\maketitle

\newpage



%
%	SECTION 1
%

\section{概要}

このレポートでは,以下に与えられる5つの課題に関しての考察,またそのプログラムを作成する過程を示すものである.ただし,プログラムはMIPSアセンブリ言語で記述し,SPIMを用いて動作を確認する.


\begin{enumerate}

\item SPIMが提供するシステムコールをC言語から実行できるようにしたい.教科書,(以下教科書はコンピュータの構成と設計(パターソン&ヘネシー)第四版を指すこととする)B.6節.「手続き呼び出し規約」に従って,各種手続きをアセンブラで記述せよ.ファイル名は,syscalls.sとすること. 

また,記述したsyscalls.sの関数をC言語から呼び出すことで,ハノイの塔(hanoi.cとする)を完成させよ.

{\baselineskip 3mm
\begin{verbatim}
void hanoi(int n, int start, int finish, int extra)
{
  if (n!=0) {
    hanoi(n-1, start, extra, finish);
    print_string("Move disk ");
    print_int(n);
    print_string(" from peg ");
    print_int(start);
    print_string(" to peg ");
    print_int(finish);
    print_string(".\n");
    hanoi(n-1, extra, finish, start);
  }
}
main()
{
  int n;
  print_string("Enter number of disks> ");
  n = read_int();
  hanoi(n, 1, 2, 3);
}
\end{verbatim}
}

spim-gccによってhanoi.sができたら,hanoi.s, syscalls.sの順にSPIM上でロードして実行し,その結果を示せ.

\item hanoi.sを例にspim-gccの引数保存に関するスタックの利用方法について,説明書.そのことは,規約上許されるスタックフレームの最小値24とどう関係しているか.このスタックフレームの最小値規約を守らないとどのような問題が生じるかについて解説せよ.


\item 以下のプログラムreport2-1.cをコンパイルした結果をもとに,auto変数とstatic変数の違い,ポインタと配列の違いについてレポートせよ.

{\baselineskip 3mm
\begin{verbatim}
int primes_stat[10];
 
char * string_ptr = "ABCDEFG";
char string_ary[] = "ABCDEFG";
 
void print_var(char *name, int val)
{
  print_string(name);
  print_string(" = ");
  print_int(val);
  print_string("\n");
}
 
main()
{
  int primes_auto[10];
 
  primes_stat[0] = 2;
  primes_auto[0] = 3;
 
  print_var("primes_stat[0]", primes_stat[0]);
  print_var("primes_auto[0]", primes_auto[0]);
}
 
\end{verbatim}
}


\item printfなど,一部の関数は,任意の数の引数を取ることができる.これらの関数を可変引数関数と呼ぶ.MIPSのCコンパイラにおいて可変引数関数の実現方法について考察し,解説せよ.

\item printfのサブセットを実装し,SPIM上でその動作を確認する応用プログラム(自由なでもプログラム)を作成せよ.

\end{enumerate}


%
%	SECTION 2
%

\section{プログラムの作成方針}


\subsection{課題1}

課題1では,与えられたハノイの塔のプログラムhanoi.cに関して,それぞれ

\begin{itemize}
  \item[(1)]print\_string()
  \item[(2)]print\_int()
  \item[(3)]read\_string()
  \item[(4)]read\_int()
\end{itemize}

の4つの関数をアセンブリで定義したsyscalls.sというプログラムを作成し,ハノイの塔のプログラムhanoi.cをクロスコンパイラを用いてMIPS用にコンパイルしたアセンブリプログラムhanoi.sと同時に読み込ませてC言語で書いたプログラムからsyscallを呼び出す,という処理を実現する.

また,出力結果は以下のように表示するようにプログラムを作成する.以下はディスクの枚数を3として実行した結果である.

{\baselineskip 3mm
\begin{verbatim}

Enter number of disks> 3
Move disk 1 from peg 1 to peg 2.
Move disk 2 from peg 1 to peg 3.
Move disk 1 from peg 2 to peg 3.
Move disk 3 from peg 1 to peg 2.
Move disk 1 from peg 3 to peg 1.
Move disk 2 from peg 3 to peg 2.
Move disk 1 from peg 1 to peg 2.
 
\end{verbatim}
}



\subsection{課題5}

課題5では,以下の2つの関数とひとつのアセンブリプログラムを作成する.処理の概要は以下の通りに定め,下記でそれぞれについて解説する

\begin{itemize}
\item[(1)]フォーマットに従い適切な出力をコンソールに表示する関数myprintf()
\item[(2)]任意のフォーマットを引数にmyprintf()関数を呼び出すmain関数
\item[(3)]適切な値をコンソールに出力する関数を定義したアセンブリプログラムsyscalls.s
\end{itemize}


(1)myprintf()関数では,与えられた引数に関して適切な形で変数の代入などを行いコンソール上に結果を出力する.
(2)main関数では,myprintf() 関数を,任意の引数とともに呼び出す.
(3)syscalls.sではmyprintf()関数内である変数の内容を表示する場合にそれぞれ適切な型でコンソールに出力する関数を複数定義するプログラム.



%
%	SECTION 3
%

\section{プログラムリストおよび,その説明}

それぞれの課題について,完成したプログラムを末尾に添付する.このセクションでは,プログラムの主な構造について説明する.

\subsection{課題1}

まず,SPIMでコンソール上に出力を表示する方法を記述する.

SPIMでオペレーティングシステム的なサービスを実行するためには,システムコール(syscall)という命令を使うことになっている.

syscallを使い,サービスを要求する手順は以下のようにまとめることができる.

\begin{itemize}
\item[1]1. レジスタ\$v0に使いたいサービスのシステムコールコードを格納
\item[2]2. 引数をレジスタ\$a0から\$a3(浮動小数点数の値は\$f12)にロード
\item[3]3. 値を返すシステムコールは結果をレジスタ\$v0(浮動小数点数の値の場合は\$f0)に収める
\item[4]4. syscallで実行
\end{itemize}

サービスを使うためには手順1で\$v0に使いたいサービスの対応するデータをintで渡さなければならない.これをシステムコールコードという.(参考文献1 P.781参照)


次に,作成したプログラムsyscalls.sの説明をする.

まずprint\_int関数から解説を行う.


関数が呼ばれた直後に,被呼び出し側でスタックフレームの領域を確保し,\$spを退避する.ここでは24バイト分確保している.

% TODO: Resource of this information
また,C言語から関数を呼び出すときには\_funcname:のように最初にアンダーバーをつけること.これはコンパイルに依存し,今回はgccを用いている. 

{\baselineskip 3mm
\begin{verbatim}

_print_int:
  subu  $sp, $sp, 24
  sw    $ra, 20($sp)
 
\end{verbatim}
}

次の2行で,各サービスに対応するシステムコールコードを\$v0に格納してsyscallを実行している.ここでは,引数は\$a0に格納されて渡されることを仕様の前提としており,それを利用している.

{\baselineskip 3mm
\begin{verbatim}

li    $v0, 1  # 1: print_int
syscall
 
\end{verbatim}
}

次の3行で,戻りアドレスを復元し,スタックフレームを破棄,復元した戻りアドレスに従ってもとの場所にジャンプ,という処理を行っている.

{\baselineskip 3mm
  \begin{verbatim}

lw    $ra, 20($sp)
addu  $sp, $sp, 24
j     $ra


  \end{verbatim}
}

他の4つの関数print\_int, read\_string, read\_intについても同様の処理を行う.各関数の定義で異なっているのはシステムコールコードの値のみである.



\subsection{課題5}

末尾に添付したプログラムmyprintf.cについて解説を行う.

main()関数では,任意のフォーマットと引数とともにmyprintf()関数を呼び出す処理を行う

{\baselineskip 3mm
  \begin{verbatim}

int main() {
  myprintf("I am %s, my age is %d", "Yuta Totsuka", 21);
  return 0;
}

\end{verbatim}
}

プログラムが実行されると,まずこのmain関数が呼ばれ,myprintf()関数を呼び出す.このサンプルでは可変引数として

\begin{itemize}
\item[1]"I am \%s, my age is \%d": string
\item[2]"Yuta Totsuka": string
\item[3]21: integer
\end{itemize}

の3つの引数を与えることとする.

このmyprintf()関数が無事実行されたらmain()関数は0を返してこのプログラムは終了する.


次に,myprintf()関数部について説明する.

まず,この関数では可変引数を受け取る必要がある.1つめはfmtをchar型のポインタで,2つ目以降は...の形で受け取っている.

\begin{verbatim}
void myprintf(char *fmt, ...) {
\end{verbatim}

以下で必要な引数を宣言している.

{\baselineskip 3mm
  \begin{verbatim}

int i, argc = 0;
char *s;

\end{verbatim}
}

一番外側のループであるwhileでは,fmt++で*fmtのポインタが指す場所をひとつずつ繰り上げていき,char *fmtの中身がなくなるまで出力する処理を繰り返す,という仕組みである.

{\baselineskip 3mm
  \begin{verbatim}

while (*fmt) {

// 出力処理

    fmt++;
}

\end{verbatim}
}

続いて上記出力処理部分では*fmtが指すデータが'\%'だった場合とそうでない場合とで出力の処理を分岐する.

{\baselineskip 3mm
  \begin{verbatim}

// 出力処理
 if (*fmt == '%') {
      // 各型に対応する変数の出力処理
    }else {
      // 文字をそのまま出力
    }
    fmt++;

\end{verbatim}
}

'\%'だった場合はその'\%'自体は出力せずにfmtを繰り上げ次のアドレスを指すようにし,そこに入っているアルファベットに従って,switchで各型に応じた出力処理に振り分ける,という処理を行う.

{\baselineskip 3mm
  \begin{verbatim}

 // 各型に対応する変数の出力処理
      fmt++;
      argc++;
      switch (*fmt) {
        case '%':
          // Process of %%
          break;
        case 'c':
          // Process of %c
          break;
        case 'd':
          // Process of %d
          i = *((int*) ((char *)&fmt + argc * sizeof(void *)) );
          print_int(i);
          break;
        case 'u':
          // Process of %u
          break;
        case 's':
          // Process of %s
          s = *((int*) ((char *)&fmt + argc * sizeof(void *)) );
          print_string(s);
          break;
        break;
      }

\end{verbatim}
}

'\%'以外の場合はその文字をそのまま出力する.

{\baselineskip 3mm
  \begin{verbatim}

      // 文字をそのまま出力 
      print_char(*fmt);

\end{verbatim}
}


また,stdio.hライブラリで実装されているprintf()関数を使わずにmyprintf()関数を実現するために,コンソールに文字を出力するための関数をsyscalls.sで定義している.各関数で行っている処理の詳細は,section3.1を参照. 




%
%	SECTION 4
%

\section{プログラムの使用例・テスト}

このセクションでは,プログラムの使用例を示しながら実際にテストを行う過程を示す.

\subsection{課題1}

hanoi.cをクロスコンパイラを用いてMIPSプラットフォーム向けのアセンブリプログラムhanoi.sを作成する. 

その後spimを用いてhanoi.s,syscalls.sという順番でプログラムをロードして実行すると,以下のように表示されハノイの塔の高さをきかれる. 

{\baselineskip 3mm
\begin{verbatim}
Enter number of disks > 
\end{verbatim}
}

ここで3を入力すると,ハノイの塔の手順が以下のようにコンソールに出力される. 

{\baselineskip 3mm
\begin{verbatim}

Enter number of disks> 3
Move disk 1 from peg 1 to peg 2.
Move disk 2 from peg 1 to peg 3.
Move disk 1 from peg 2 to peg 3.
Move disk 3 from peg 1 to peg 2.
Move disk 1 from peg 3 to peg 1.
Move disk 2 from peg 3 to peg 2.
Move disk 1 from peg 1 to peg 2.
 
\end{verbatim}
}





\subsection{課題5}

クロスコンパイラを用いて以下のようにmyprintf.cをmyprintf.sにコンパイルする. 

\begin{verbatim}
$ spim-gcc myprintf.c
\end{verbatim}


myprintf.s,syscalls.sの順にプログラムをspimでロードし実行すると以下のような出力を得ることができる.

{\baselineskip 3mm
\begin{verbatim}
I am Yuta Totsuka, my age is 21
\end{verbatim}
}




%
%	SECTION 5
%

\section{プログラム作成における考察}

各課題において,プログラム作成過程における考察を以下に示す.

\subsection{課題1}

関数名に\_funcname:のように最初にアンダーバーをつける理由について考察したい.

C言語プログラムからsyscalls.sにおけるpinrt\_int()などの関数を呼び出す場合には,関数名の最初にアンダーバーをつけることが決まりになっている.

だがC言語のプログラムをアセンブリのプログラムに変換した場合結局は.sファイルにコンパイルした後に2つのプログラムを読み込むことになるので,2つのアセンブリプログラムを読み込むこととかわりはないように思われる.よってこの決まりは,外部プログラムを呼び出す全般に置いて,アンダーバーをつけるのがMIPSアセンブリの決まりであるのかもしれない.

これはMIPSが複数のプログラムを結びつけるときに内部にある関数と外部にある関数を判断するためにつけられるものであると考察した.


\subsection{課題5}

標準のC言語ライブラリにおけるprintf()関数でも実装されている機能としてフォーマット指定子として"\%\%"が与えられた時に出力として"\%"を表示する機能を実装した.実装部分はセクション7のプログラムリストmyprintf.cにおける以下の部分である.

\begin{verbatim}
        case '%':
          // Process of %%: print '%'
          print_char(*fmt);
          break;
\end{verbatim}

ここでは,syscalls.sにおいて定義しているprint\_char()関数を用い,*fmtに格納されている"\%"を出力させている.

\subsubsection{疑問点1}

なぜ解説プログラムでは,Cソース中でprint\_char()関数を定義しているのか.MIPSアセンブリのsyscallのサービスとして提供されているものを使用した時との違いは何か.

myprintf.cでは,syscalls.sにおいて定義したprint\_char()関数を呼び出すように実装し,期待取りの出力を得ることが出来た.Cソース中でprint\_char()関数を定義すると何か違いが生まれるのか.また,アセンブリにおいて文字1文字を表示させたい場合は違う方法を取った方がよいのだろうか.




%
%	SECTION 6
%

\section{得られた結果に関する,あるいは試問に対する回答}


各課題において,得られた結果に関する,あるいは試問に対する回答を以下に示す.
また,結果に対してのいくつかの疑問点とそれについての考察を記述する.


% \subsection{課題1}

\subsection{課題2}

% hanoi.sを例に,spim-gccの引数保存に関するスタックの利用方法について,説明せよ.そのことは,規約上許されるスタックフレームの最小値24とどう関係しているか.このスタックフレームの最小値規約を守らないとどのような問題が生じるかについて解説せよ.

hanoi.sにおいて,hanoi関数の冒頭の部分について解説する.

まず以下で24バイト分スタックフレームを生成し,戻りアドレス,フレームポインタを格納し,古いフレームポインタを退避させている.

\begin{verbatim} 
_hanoi:
	subu	$sp,$sp,24
	sw	$ra,20($sp)
	sw	$fp,16($sp)
	move	$fp,$sp
\end{verbatim}

だが,その直後で第一引数はその領域を越えて第一引数から第四引数を格納していることが分かる.

\begin{verbatim} 
	sw	$a0,24($fp)
	sw	$a1,28($fp)
	sw	$a2,32($fp)
	sw	$a3,36($fp)
\end{verbatim} 

ここで,2つの疑問点が浮上した.

まず,生成したスタックの領域を越えてデータを保存することは,プログラムがエラーを起こす原因になるのではないか.

これは,gccコンパイラの仕様である.gccでは24バイトが最小スタックフレーム幅として設定されており,これは呼び出し側が渡す引数(第一引数から第四引数)を,呼び出し側のスタックフレームに格納するからである.

次に,MIPSの仕様(手続き呼び出し規約)として第一引数から第四引数はスタックに格納せずに非呼び出し側関数に渡すのではなかったか.

この疑問点に関しては,実質は非呼び出し側のスタックフレームには保存されていない.これがこのような挙動を行う利点である.なぜかというと,非呼び出し側からみると,第五引数以降は第四引数までとスタック上で連続することになるので,コンパイラ側でプログラムの変換がしやすいからである.




\subsection{課題3}

report2-1.cをMIPSプラットフォーム向けにコンパイルしたプログラムreport2-1.sを読むことで,C言語におけるauto変数とstatic変数を,アセンブリでどのように実現しているのかを理解する.

mips-gccでコンパイルしたreport2-1.sはセクション7を参照.


ソースに従い,まずstaticについての説明を行う.

report2-1.sプログラムの最後に以下のような記述がある.

\begin{verbatim}
 .comm   _primes_stat,40
\end{verbatim}

これは以下と同じ命令である.

\begin{verbatim}
 .data
    .align 2
    _primes_start:
        .space  40
\end{verbatim}

ここでは,データセグメント内に,\_primes\_startというラベルをつけて,データを40バイト確保している.

つまりこのときに長さ10の配列のために確保された40バイトは,この配列を使うためだけに確保されている.よってこれは,プログラムの開始から終了時まで,値を保持しつづけるという性質を持つことになる.

だがあくまでこれは規約上であり絶対に上書きされないということはない.関数などが再帰的に呼び出された場合は,その領域を上書きすることがある.

その領域しか使わないからこそ,再帰的な処理の時などのときの処理は煩雑になりそうである.


次にmain部分にフォーカスして読んでいく.

まず64バイト分のスタックフレームを生成し,戻りアドレス\$raとフレームポインタ\$fpを更新.

\begin{verbatim}
main:
    subu    $sp,$sp,64
    sw  $ra,60($sp)
    sw  $fp,56($sp)
\end{verbatim}

その後フレームポインタを\$spと同じアドレスを指すようにする.

上で一度\$fpを更新してもう一度ここで更新する意味が理解できないが,それは後の疑問点で考察する.

\begin{verbatim}
 move    $fp,$sp
\end{verbatim}

次に各配列の先頭の値を準備する.

まずはstatic変数側の配列の先頭の値を準備する.

ここでは,\$v0の値を\_primes\_startラベルで確保されてるデータセグメントの最初に格納していることに注意.


\begin{verbatim}
 li  $v0,2           # 0x2
    sw  $v0,_primes_stat
\end{verbatim}

次はauto変数に対応する部分についてみていく.16(\$fp)ということで,スタックフレームのデータ4個分のところに値を格納している.

なぜ16なのかは,疑問点にて考察する.

\begin{verbatim}
 li  $v0,3           # 0x3
    sw  $v0,16($fp)
\end{verbatim}

出力する予定の文字列を読み込んで,さっき格納したstaticの配列の一番最初の値を持ってきて,出力するためのルーチンを呼ぶ.

\begin{verbatim}
 la  $a0,$LC3
    lw  $a1,_primes_stat
    jal _print_var
\end{verbatim}

今度はautoでも同じことをする.

\begin{verbatim}
 la  $a0,$LC4
    lw  $a1,16($fp)
    jal _print_var
\end{verbatim}

スタックを解放してmain部終了.

\begin{verbatim}
 move    $sp,$fp
    lw  $ra,60($sp)
    lw  $fp,56($sp)
    addu    $sp,$sp,64
    j   $ra
\end{verbatim}

ここからわかるように,auto変数ではstatic変数と違ってスタックで配列を管理していることが分かる.

つまり,この関数が終了すれば\$spに確保したバイト数が足され,スタックが解放され,その結果保持していたauto変数の配列も消えてしまう.

ここから,アセンブリ言語では,C言語のstatic変数をメモリに別の領域として確保して扱い,auto変数をプログラム中の関数のスタックに格納することによって,両者の性質を実現していることが分かる.


また、アセンブリでC言語における配列とポインタの扱い方の違いについても考察してみたい。

C言語において配列とポインタは同等のものとして扱うことが出来る。では、アセンブリにおいてこの二者の扱い方の違いはあるのだろうか。

以下のtest.cというプログラムを用意し、それをコンパイルしたtest.sというMIPSアセンブリプログラムを読みながら挙動の違いについて確認してみる。

\begin{verbatim}
// test.c
char * string_ptr   = "String of pointer";
char   string_ary[] = "String of Array";
\end{verbatim}

まず注目すべきは領域の確保の仕方である。

asciizで文字列を格納している部分は同じであるが、ポインタの方は\$LC0というラベルを使用し、\_string\_ptrにおいて.wordを使ってさきほどのラベルを指定して文字列を格納している。

\begin{varbatim}
$LC0:
  .asciiz "String of pointer"
  .data
  .align  0
  .align  2
_string_ptr:
  .word $LC0
  .align  2
\end{varbatim}

一方\_string\_aryに関しては、そのままそのラベル部分で領域を確保して文字列を格納している。

\begin{varbatim}
\_string\_ary:
  .asciiz "String of Array"
  .rdata
  .align  0
  .align  2
\end{varbatim}

この違いは、一体何を生むのだろうか。

ポインタでは、\_string\_ptrの示すアドレス(文字列の先頭番地のなるのか)を変えることで、容易に他のデータを指し示すよう変更を加えることができる。文字列ではそのラベル自体が文字列をさしているのでそのような変更を行うことはできない。これは間接的にデータを指し示すポインタを用いるひとつの利点になり得るだろう。

C言語のレベルに落とし込むとさらに疑問は生まれる。

C言語内でもstring\_ptrの値を直接アドレスで上書きすれば他のデータの先頭を指し示すよう変更を加えることができることは容易に想像できる。

だが同様にC言語ソース中でstring\_aryはデータの先頭を指し示すアドレスであるはずなのでこれを書き換えればstring\_ary[]内のデータも変更されるような気がする。だがそのような操作を行う場合、アセンブリプログラムでは、もう一度ラベルの中のデータを塗り替えてしまうぐらいしか方法が思いつかない。ということは、C言語でこのような操作を行うことはできないのだろうか。

\subsection{課題4}

% printfなど,一部の関数は,任意の数の引数を取ることができる.これらの関数を可変引数関数と呼ぶ.MIPSのCコンパイラに置いて可変引数関数の実現方法について考察し,解説せよ.

C言語において,ドット3つにおいて可変引数を取り出すために以下のような試行を行った.

まず,以下のようなふたつのプログラム,main.cとmyprintf.cを作成する.

\begin{verbatim}
# main.c

#include
#include "myprintf.c"
 
int main() {
  char string[] = "Second Argument";
  myprintf("Argument: %s", string);
  return 0;
}
\end{verbatim}

\begin{verbatim}
# myprintf.c

void myprintf(char *fmt, ...) {
  printf("Format =>  %s\n", fmt);
  printf("Address1 =>  %d\n", &fmt);
  char *p = ((char*)&fmt) + 4;
  printf("Address2 =>  %d\n", p);
  printf("Second Argument => %s\n", p);
}

\end{verbatim}

ここで,main.cをコンパイルして実行すると,以下のような出力を得る.

\begin{verbatim}
bash-4.1$ gcc main.c 
bash-4.1$ ./a.out 
Format =>  Argument: %s
Address1 =>  1365075928
Address2 =>  1365075932
Second Argument => 
\end{verbatim}

得られた出力に関して解説を行う.

まず,第一の

\begin{verbatim}
Format => Argument: %s
\end{verbatim}

より,第一引数の文字列(フォーマット)がきちんと渡っていることが分かる.また,その次の行の

\begin{verbatim}
Address1 =>  1365075928
\end{verbatim}

からは,\&fmtが生成した文字列の最初のアドレスをきちんと指していることがわかる.そして

\begin{verbatim}
char *p = ((char*)&fmt) + 4;
\end{verbatim}

で*pというポインタを生成した.これによりpはfmt(ポインタになっている)+4バイト先のレジスタ,つまり32bitコンピュータの1データ分下に格納されていることになっている.これは,その次の行,

\begin{verbatim}
printf("Address2 =>  %d\n", p);
\end{verbatim}

において

\begin{verbatim}
Address2 =>  1365075932
\end{verbatim}

が出力されることによりAddress1 + 4 = Address2になっていることを確認できる.

ここで,Second Argumentという文字列を第二引数として渡したい.

\begin{verbatim}
第2引数は,アドレス = (旧$sp +04) からの 4バイトに格納されている事が分かります.
参照:http://www.swlab.cs.okayama-u.ac.jp/~nom/lect/p3/what-is-va-arg.html
\end{verbatim}

これより,可変引数の場合スタックで第一引数の次にポップされることになるなのでここで*pは第二引数のアドレスを指しているはずであるので,*pが指しているデータを確認してみる.

\begin{verbatim}
Second Argument => 
\end{verbatim}

だが出力は以上のようになり,*pが指しているデータが空であることを示している.

環境の違いの理由(32bitか64bitか)も考慮した試行を行ったので環境の違いが原因ではないようである.

printf()関数がこのような表示の仕方に対応していないのではないかと推測した.


\subsection{課題5}


フォーマット"\%x"および"\%o"の実装について

フォーマット"\%x"で,与えられた整数を16進数で出力し,フォーマット"\%o"で与えられた整数を8進数で出力する機能の実装において,うまく出力されなかった理由を考察する.

該当箇所はソース中25行目から58行目までである.

16進数の場合についてまず解説を行う.

最初に引数を受け取り,必要な文字列の配列を確保する.

\begin{verbatim}
        case 'x':
          i = *((int*) ((char *)&fmt + argc * sizeof(void *)) );
          str = (char *) malloc(15);
\end{verbatim}

次に,16進数に変換するアルゴリズムを適用する.ここでは,引数を16でわり,そのあまりをstrに格納している.ポイントは,16進数における0~9の場合とA以降の場合で条件分岐を行い,ASCIIコードの対応する数字を代入することである.

\begin{verbatim}
          while(i) {
            str = i % 16;
            if(10>*str){
              *str+=0x30;     //0x30は0の文字コード
            }else{
              *str+=(0x41-0x0A);  //0x41はAの文字コード
            }
\end{verbatim}

次に,iを16で割ったものをiに代入し,iがゼロになるまでこの処理を繰り返す.

\begin{verbatim}
          i/=16;
          }
\end{verbatim}

最後に文字列を出力して終了する.

\begin{verbatim}
          print_string(str);
          break;
\end{verbatim}

このプログラムでは結果は出力されない.アルゴリズムの部分またはポインタの部分で間違いをしている可能性が高いが,結果を探ることができなかった.



\newpage

%
%	SECTION 6
%

\section{作成したプログラムのソースコード}

それぞれについて,作成したプログラムは以下である.

\subsection{課題1}

\lstinputlisting[caption=hanoi.s, label=hanoi.s]{../../Second/hanoi.s}

\lstinputlisting[caption=syscalls.s, label=syscalls.s]{../../Second/syscalls.s}


\subsection{課題5}

\lstinputlisting[caption=myprintf.c, label=myprintf.c]{../../myprintf/myprintf.c}

\lstinputlisting[caption=syscalls.s label=syscalls.s]{../../myprintf/syscalls.s}

%
% SECTION 7
%

\section{参考文献}

\begin{enumerate}
\item コンピュータの構成と設計 第4版 上・下 ジョン・L. ヘネシー (著), デイビッド・A. パターソン (著), 成田 光彰 (翻訳)
\item Computer Organization and Design, Fifth Edition: The Hardware/Software Interface (The Morgan Kaufmann Series in Computer Architecture and Design) by David A. Patterson (Author), John L. Hennessy  (Author)
\end{enumerate}


\end{document}

